\documentclass{article}
\usepackage{graphicx} % Required for inserting images
\usepackage[T1]{fontenc}
\usepackage[utf8]{inputenc}
\usepackage{hyperref}
\usepackage[english,serbian]{babel}
\usepackage[unicode]{hyperref}

\title{Minimal Sorting by Reversals}
\author{
Vuk Stefanović 66/2019, Anja Čolić 231/2019}

\date{Septembar 2023}
\newcommand{\fakultet}{Matematički fakultet}
\newcommand{\kurs}{Seminarski rad u okviru kursa Računarska inteligencija}



\begin{document}


\pagenumbering{gobble}

\maketitle


\begin{center}
\kurs
\end{center}
\begin{center}
\fakultet
\end{center}


\newpage
\pagenumbering{arabic}
\renewcommand{\contentsname}{Sadržaj}
% Druga stranica - Sadržaj
\tableofcontents
\newpage

\





\maketitle

\section{Uvod}

Minimum Sorting by Reversals je pojednostavljena verzija problema preuredjivanja genoma koja pokušava otkriti evolutivni odnos izmedju različitih genoma i jedan je od mnogih izazovnih problema u bioinformatici. Rešavanje ovog problema optimalno je dokazano da je NP-teško, i stoga su razvijeni različiti aproksimativni algoritmi. Mi ćemo u radu za rešavanje koristiti genetski algoritam i uporediti rezultate dobijene i grubom silom.

\section{Formulacija problema}
Na ulazu imamo permutaciju brojeva od 1 do n. Potrebno je permutaciju dovesti do permutacije identiteta (permutacija sortirana rastuće), tako da broj operacija bude minimalan.
\newline
\newline
Dozvoljene operacije su operacije invertovanja intervala [i,j], 1<=i,j<=n.
\newline
\newline
Primer:
\newline
Ulazni niz: [2,1,4,3,5]
\newline
Koraci:
\newline
[2,1,4,3,5] → invertujemo [0,1]
\newline
[1,2,4,3,5] → invertujemo [2,3]
\newline
[1,2,3,4,5]
\newline
\newline
Na izlazu dobijamo sortiran niz, koristeći dve inverzije.

\newpage
\section{Ideja rešenja}
Svaka jedinka u populaciji prestavlja niz inverzija koje primenjujemo na početnu permutaciju. Pošto broj inverzija nije unapred poznat svaka jedinka može biti različite dužine (uvek parne dužine).

\subsection{Fitnes funkcija}

Za izračunavanje fitnes funkcije, pored broja inverzija, bitan nam je i broj prelomnih tačaka niza (mesta na kojima niz prestaje da bude rastući).
\newline
\newline
Kažemo da između dva susedna elementa niza postoji prelomna tačka ako je apsolutna razlika tih elemenata različita od 1.
\newline
\newline
Primer:
\newline
[1,2,4,5,3] → prelomne tacke su između 2 i 4, kao i 5 i 3
\newline
\newline
Fitnes funkciju definišemo kao:
\newline
\newline
A*dužina(hromozoma)+B*brojPrelomnihTacaka
\newline
\newline
A i B su proizvoljni parametri koji određuju uticaj broja prelomnih tačaka i dužine hromozoma.
\newpage
\subsection{Populacija}
Veličina populacije je n*log(n), gde je n dužina niza koji se sortira.
Dužina hromozoma se kreće od broja tačaka preloma do 3n. Delimo populaciju na dva dela, na oba dela primenjujemo selekciju, mutaciju i ukrstanje, vodeci računa o tome da ne izgubimo najbolje jedinke u narednoj generaciji ( Elitizam ).

\textbf{Komentar :}  Velicina populacije preuzeta iz rada \cite{prva}

\subsection{Selekcija}
Koristićemo turnirsku selekciju.

\subsection{Ukrštanje}
U zavisnosti da li su oba roditelja iste dužine, razlikovaćemo 2 vrste ukrštanja:
\begin{itemize}
\item 
Ako su iste dužine - Koristimo dvopoziciono ukrštanje.
\item 
Ako su različite dužine - Koristimo takodje dvopoziciono ukrštanje ali gde se početna pozicija većeg hromozoma
odabere nasumično i onda se ceo manji hromozom stavi na poziciju nakon nje (apsorpcija).
\end{itemize}
\newline
\newline
Apsorpcija primer:
\begin{itemize}
    \item 
    Roditelj 1 : [0, 1, 2, 4 , 0, 3, 1, 2]
    \item 
    Roditelj 2 : [0, 3, 0, 2]
    \item 
    Prva izabrana pozicija = 2
    \item 
    Dete 1 : [0, 1, 0, 3, 0, 2, 1, 2 ]
    \item 
    Dete 2 : [2, 4 , 0, 3]

\end{itemize}

\newpage

\subsection{Mutacija}

Na svaku od jedinki primenjujemo jedan od 4 operatora mutacije koji random određujemo.
\newline
\newline
Operatori mutacije:
\begin{itemize}
\item
 Menjamo nasumično izabran gen 
\item
 Izaberemo 2 pozicije  i zamenimo vrednosti na tim pozicijama novim random vrednostima u opsegu od 0 do veličina niza.

\item
 Odaberemo 2 inverzije [x1, x2] i [y1, y2] i na pozicije nakon x2 i pre y1 umetnemo 2 random broja čime
dobijamo dužu jedinku.
\item
Odaberemo 2 inverzije [x1, x2] i [y1, y2], obrisemo x2 i y1 i time dobijemo novu jedinku

\end{itemize}

\textbf{Komentar}: Ideja za poslednje dve mutacija uzeta iz rada \cite{prva}

\newpage
\section{Modifikacije genetskog algoritma}
Uočili smo  problem koji bi mutacije koje smo preuzeli iz rada teško rešile:
\begin{itemize}
\item  
U situaciji kad je niz skoro sortiran (kad je potrebno uraditi samo jednu ili dve inverzije) poslednje dve mutacije ne mogu da reše ovu situaciju, zato uvodimo mutaciju 1.
\end{itemize}
\newline


Zbog situacije da niz na kraju često nije bio sortiran, izvršavali smo ponovno pokretanje genetskog algoritma nad  datim nizom. Nakon svakog pokretanja novi kod najbolje jedinke dodavali smo na stari.
\newline
\newline
Na primer:
Početni niz: [3, 2, 1, 5, 4]
\newline
Algoritam vrati niz :[1, 2, 3, 5, 4]
\newline
Kod najbolje jedinke : [0, 2]
\newline
Ponovo pozovemo algoritam za niz [1 ,2 ,3, 5, 4], algoritam vrati kod najbolje jedinke [3, 4], i to nadovežemo na [0, 2]



\newpage








\section{Rezultati}\label{sec:rez}
Upoređujemo algoritam grube sile i genetski algoritam samo za nizove najviše dužine pet, zbog velike složenosti grube sile i činjenice da se genetski algoritam mnogo brže izvršava za duže nizove.
\newline
\newline
Nastavljajući, upotrebljavali smo tri različita genetska algoritma za dužine nizova  najviše 15, jer smo primetili da je za duže nizove bilo potrebno znatno više vremena za izvršavanje.

Modifikacije genetskih algoritama koje smo upotrebljavali su:
\begin{itemize}
\item 
"MSR1234" - svaka jedinka izvršava jednu random mutaciju od prethodno opisane četiri.
\item
"MSR234" - svaka jedinka izvršava  jednu random mutaciju od tri koje predstavljaju unapred izabran podskup prethodno četiri.
\item 
"MSR2" - svaka jedinka izvršava treću po redu opisanu mutaciju sa odredjenom verovatnoćom.
\end{itemize}
\newline
\newline
Svi rezultati su prikazani u narednoj tabeli.
\newline


\begin{table}[htbp]

\centering
\small 

\label{tab:my-table}
\begin{tabular}{|c|c|c|c|c|c|c|c|c|c}
\hline
N & Array  & Brute Force  & "MSR1234" & "MSR234" & "MSR2"& Time \\
\hline
2 & [1, 2] & 0& 0 & 0 & 0 & 0.05 min \\
\hline
3 & [3, 1, 2] & 2& 2 & 2 & 2 & 0.05 min \\
\hline
4 &  [4, 3, 2, 1] & 1& 1 & 1 & 1 &  0.1 min\\
\hline
5  & [1, 2, 3, 5, 4] & 1 & 1 & 1 & 1 & 0.1 min\\
\hline
6  & [5, 4, 1, 2, 3, 6] & - & 2 & 2 & 2 &  0.2 min\\
\hline
7  & [2, 6, 1, 4, 7, 5, 3] & - & 6 & 6 & 6 & 0.2 min\\
\hline
8 & [6, 2, 1, 5, 3, 8, 4, 7] & - & 5 & 6 & - & 0.2 min \\
\hline
9  & [9, 5, 4, 2, 1, 3, 6, 7, 8] & -  & 5 & 6 & 5 &  0.5 min \\
\hline
10 & [8, 3, 6, 1, 9, 10, 4, 2, 7, 5] & - & 7 & 8 & 7 &  0.5 min\\
\hline
11  &  [3, 10, 5, 4, 11, 7, 2, 6, 9, 1, 8] & - & 9 & 9 & 11 &  1 min\\
\hline
12 & [5, 7, 11, 3, 9, 2, 12, 4, 10, 1, 6, 8] & - & 10 & 9 & 11 & 1 min\\
\hline
13& [4, 1, 2, 5, 3, 6, 7, 10, 8, 9, 11, 12, 13] & - & 7 & 5 & 5 & 1 min\\
\hline
14 & [4, 1, 2, 5, 3, 6, 7, 9, 8, 10, 11, 14, 12, 13] & - & 10 & 6 & 14 &  1 min\\
\hline
15& [4, 1, 2, 5, 3, 6, 7, 9, 8, 10, 11, 12 , 15, 14, 13] & - & 5 & 5 & 7 &  1 min\\

\hline
\end{tabular}
\caption{Rezultati koji prikazuju broj inverzija dobijenih primenom svakog algoritma na random generesine permutacije dužine od 2-15}
\end{table}



\newpage

\section{Zaključak}\label{sec:rez}
U navedenom radu smo predstavili pristup za rešavanje problema minimalnog sortiranja inverzijom intervala. Implementirali smo ovaj pristup pomoću genetskog algoritma sa promenljivom dužinom hromozoma.
\newline

Važno je napomenuti da naše rešenje ne može garantovati optimalno rešenje problema. Na kraju izvođenja algoritma, dobijamo broj inverzija potrebnih za sortiranje niza, kao i listu tih inverzija koje treba primeniti da bi se postigao sortirani niz.
\newline

Moguće je pokrenuti program više puta za istu instancu problema (isti početni niz), i ovo može rezultirati različitim rešenjima. Međutim, najmanji broj inverzija među svim dobijenim rezultatima smatramo najboljim rešenjem koje smo do sada pronašli.
\newline
Važno je naglasiti da nije moguće precizno odrediti koliko se naše rešenje razlikuje od optimalnog. Praktično, da bismo saznali koliko smo daleko od optimalnog rešenja, morali bismo da pronađemo baš to optimalno rešenje, što može biti izazovno ili nemoguće.
\newline
\newline
\textbf{Objasnjenje za prethodno}
Ako program vrati broj inverzija n i istovremeno imamo informaciju da je naše rešenje udaljeno za k od optimalnog rešenja, tada možemo zaključiti da je optimalno rešenje n - k inverzija. Ovo je zato što znamo da je naše rešenje od n inverzija bolje od rešenja od n + k inverzija (jer je bliže optimalnom rešenju za k).
\newline
\newline
Za rešenje od n + k inverzija, možemo pretpostaviti da ono nije optimalno, jer već imamo rešenje od n inverzija koje je "optimalnije" (bliže optimalnom rešenju). Dakle, možemo zaključiti da rešenje od n + k inverzija ne može biti optimalno, jer imamo bolje rešenje sa samo n inverzija.
\newline

Ova analiza nam pomaže da relativno procenimo kvalitet našeg rešenja u odnosu na optimalno rešenje, bez potrebe da stvarno pronađemo optimalno rešenje.
\newpage

\textbf{Dalji tok unapredjivanja}
U daljem razvoju algoritma, razmotrenje različitih operacija mutacije i ukrštanja svakako može doprineti poboljšanju performansi algoritma. Promenom ovih operatora možemo eksperimentisati sa različitim strategijama i prilagoditi ih specificiranom problemu kako bismo postigli bolje rezultate.
\newline
Takođe, razmatranje različitih strategija selekcije, kao što je selekcija rangiranjem, takođe može biti korisno. Različite strategije selekcije mogu imati različite efekte na diverzitet populacije i konvergenciju algoritma.
\newline

Ako je vreme izvršavanja ključni faktor, prelazak na programski jezik poput C ili C++ može ubrzati izvođenje algoritma. Međutim, važno je napomenuti da ovo neće promeniti asimptotsku složenost algoritma. Takođe, optimizacija algoritma i struktura podataka može takođe doprineti poboljšanju brzine izvršavanja.
\newline

U svakom slučaju, iterativno eksperimentisanje sa različitim komponentama algoritma i evaluacija njihovog uticaja na performanse može dovesti do boljeg rešenja za dati problem.

\newpage

\addcontentsline{toc}{section}{Literatura}
		\appendix
		
		\iffalse
		\bibliography{seminarski} 
		\bibliographystyle{plain}
		\fi
		
		\begin{thebibliography}{9}
			
			\bibitem{prva}
                \href{https://www.researchgate.net/publication/221009015_Sorting_unsigned_permutations_by_reversals
_using_multi-objective_evolutionary_algorithms_with_variable_size_}{Sorting Unsigned Permutations by Reversals using Multi-Objective EvolutionaryAlgorithms 
with Variable Size Individuals, Ahmadreza Ghaffarizadeh, Kamilia Ahmadi and Nicholas S. Flann}
			
		
			
                

               
                
		\end{thebibliography}









\end{document}
